\documentclass[a4paper,10pt]{article}

\usepackage{times}
\usepackage[english]{babel}
\usepackage[latin1]{inputenc}
\usepackage[T1]{fontenc}

%adding hyperlinks to the PDF
\usepackage[pdftex, pdfstartview=FitH]{hyperref}
\hypersetup{
   colorlinks,
   citecolor=black,
   filecolor=black,
   linkcolor=black,  
   urlcolor=black
}

\author{Jean-S{\'e}bastion Gay and Ghislain Charrier} 
\title{Simbatch Documentation}
\date{\today
\\
\{jean-sebastien.gay, ghislain.charrier\}@ens-lyon.fr
\\
This is just a rough draft of what will be the
  documentation, but it is better than nothing.}

%%%%%%%%%%%%%%%%%%%%%%%%%%%%%%%%%%%%%%%%%%%%%%%%%%%%%
%% Beginning of the document
%%%%%%%%%%%%%%%%%%%%%%%%%%%%%%%%%%%%%%%%%%%%%%%%%%%%%
\begin{document}
\maketitle
\tableofcontents
\listoffigures
\listoftables

%% Global presentation of simbatch
\section{Introduction}

Grids have become a tool widely used in computer science. They permit
to compute a huge amount of tasks in a reasonable time. When computer
scientists try to developp new algorithms for the grid, one of their
main problem is the difficulty to reproduce results. The aim of
simgrid \cite{simgrid} is to provide an efficient and powerful grid
simulator.

An aspect of simgrid that is not taken into account is the batch
reservation systems. Simbatch is a C API build on top of simgrid. The
API is designed to simulate batch reservation tools execution in a
Grid environment.


%% Compiling simbatch
\section{Installing Simbatch}
\label{sec:installation}

\subsection{Dependencies}
\label{subsec:dependencies}

To compile Simbatch, you will need to have:

\begin{itemize}
  \item a *nix environment (Simbatch was tested on Linux and Mac OS
    but should work on any *nix environment)
  \item simgrid >= 3.1 (Simbatch is build over simgrid so it is
    obviously a dependency)
  \item gcc >= 4.0 (to compile the project)
  \item cmake >= 2.4 (to create the makefiles)
  \item libxml2 (used to parse input files)
\end{itemize}

\subsection{Compiling and installing Simbatch}
\label{subsec:compilation}

To install Simbatch, the first thing you need to do is to download the
sources. you can go on \url{http://graal.ens-lyon.fr/simbatch} to
download an old release. If you want a more up-to-date version, you
need to download the sources from the Simgrid svn in the contrib
version. Then, create a build directory to work with cmake. then you
can compile and install Simbatch.

\verb+ccmake+ is used instead of cmake because it is a graphical
version of the tool. The first screen you will see will tell you that
the cache is empty. This is normal, so just quit by hitting \emph{e}.
On the next screen, you can change different options. The options will
be details in section \ref{subsec:options}. Hit the \emph{c} key to
test if all needed packages are present. if everything is OK, you can
press \emph{g} to generate the makefiles. Finally, just compile as
usual with make and make install.

{\small
\begin{verbatim}
$>svn checkout svn://scm.gforge.inria.fr/svn/simgrid/contrib/Simbatch
[...]
$>cd Simbatch
$>mkdir build
$>cd build
$>ccmake ..
$>make
[...]
$>make install
[...]
$>
\end{verbatim}}

\subsection{Cmake options}
\label{subsec:options}

When you are in the \verb+ccmake+ GUI, the options you may use are:

\begin{description}
  \item [CMAKE\_INSTALL\_PREFIX] lets you choose the installation
    folder. It is not used currently.
  \item [DEBUG] activates some flags and prints more informations
  \item [GANTT] activates the
    paj{\'e}\footnote{\url{http://www-id.imag.fr/Logiciels/paje/}}
    output to visualize gantt charts. It does not work well for now.
  \item [LIBRARY\_OUTPUT\_PATH] lets you choose where the libraries
    will be put. If you let it empty, the libraries will be places in
    the lib directory where you downloaded the sources.
  \item [LOG] creates a file when executing an example with what
    appens in the batch. For example, it can display: task 1 arrived
    on batch at time 10.
  \item [NDEBUG] does something
  \item [OPT] does something
  \item [OUTPUT] is used to obtain the result of a batch in a
    file. The file contains the list of jobs the batch executed, their
    arrival time, when they started, the completion time and the
    diration.
  \item [PARANOID] activates a lot of compilation flags.
  \item [SIMGRID\_HOME] specifies the folder where simgrid is.
  \item [VERBOSE] displays more informations during an execution.
  \item [WARN] activates the basic warning flags for compilation.
\end{description} 

\subsection {Testing the installation}
\label{subsec:testing}

Go in example/batch and try to run the example. If there is no error,
your installation is ready.

{\small
\begin{verbatim}
$>cd Simbatch/example/batch
$>make
$>./batch -f simbatch.xml
[...]
$>
\end{verbatim}}


%% Input files
\section{Input files}
\label{sec:input}

Several types of files can be taken as input of a simbatch
program. Every program that you run needs to have at least 3
files. The Simbatch file, and two files used by simgrid. All these
files are described in section\ref{subsec:description}.

Simbatch can also read \verb+wld+ and \verb+swf+ files. These files
are described in section \ref{subsec:jobsDESC}. In those files, jobs
that are executed are described.

\subsection{Description files}
\label{subsec:description}

The description files are describing the batchs, the hardware and the
deployment of the processes on the hardware.

\subsubsection{Simbatch file}
\label{subsubsec:SBfile}

This XML file is quite simple. First, you give the simgrid files
(deployment and platform). Then, the batchs are described. You can
describe more than one batch. The example bellow displays a simualtion
with 2 batchs. The first one uses the Conservative Backfilling
algorithm to allocate resources. It has an internal load (so we need
to give the parser of the load). The second batch uses First Come
First Serve to allocate resources to jobs, and is dedicated (no load
is given).

{\small
\begin{verbatim}
<config>  
    <global>
        <file type="platform">platform.xml</file>
        <file type="deployment">deployment.xml</file>
    </global>
    <batch host="Batch0">
        <plugin>libfcfs.so</plugin>
        <load>1.wld</load>
        <parser>libwld.so</parser>
        <priority_queue>
            <number>3</number>
        </priority_queue>
    </batch>
    <batch host="Batch1">
        <plugin>libcbf.so</plugin>
        <priority_queue>
            <number>3</number>
        </priority_queue>
    </batch>
</config>
\end{verbatim}}

\subsubsection{Platform file}
\label{subsubsec:PFfile}

The platform file describes the hardware of the simulation. It is the
old version of platform description in Simgrid. The example below
describes a platform with 2 batchs. The first batch has 2 nodes and
the second one has just one node. The description is done by
describing CPUs, network links capacities and routes.

{\small
\begin{verbatim}
<?xml version='1.0'?>
<!DOCTYPE platform_description SYSTEM "surfxml.dtd">
<platform_description version="1">
  <!-- Cluster 0 -->
  <cpu name="B0" power="3865000"/>
  <cpu name="N0" power="3865000"/>
  <cpu name="N1" power="3865000"/>

  <network_link name="LB0" bandwidth="498000" latency="0.0"/>
  <network_link name="BW0" bandwidth="1000000" latency="0.0"/>
  <network_link name="BW1" bandwidth="1000000" latency="0.0"/>
  
  <route src="B0" dst="B0"><route_element name="LB0"/></route>
  <route src="B0" dst="N0"><route_element name="BW0"/></route>
  <route src="B0" dst="N1"><route_element name="BW1"/></route>
  <route src="N0" dst="B0"><route_element name="BW0"/></route>
  <route src="N1" dst="B0"><route_element name="BW1"/></route>

  <!-- Cluster 1 -->
  <cpu name="B1" power="3865000000"/>
  <cpu name="N2" power="3865000000"/>
  
  <network_link name="LB1" bandwidth="4980000" latency="0.0"/>
  <network_link name="BW2" bandwidth="1000000" latency="0.0"/>
  
  <route src="B1" dst="1"><route_element name="LB1"/></route>  
  <route src="B1" dst="N2"><route_element name="B2"/></route>  
  <route src="N2" dst="B1"><route_element name="B2"/></route>
</platform_description>
\end{verbatim}}

\subsubsection{Deployment file}
\label{subsubsec:DEPfile}

The deployment file describes the Simgrid processes that will be used,
and where each process will be executed. For example, the process
\verb+SB_batch+ will run on the host B0 and will take 1 argument. The
arguments have to contain the list of nodes of the batch.

{\small
\begin{verbatim}
<?xml version='1.0'?>
<!DOCTYPE platform_description SYSTEM "surfxml.dtd">
<platform_description version="1">
    <process host="Client" function="SB_client" >
        <argument value="B1" />
    </process>
    <process host="B0" function="SB_batch">
        <argument value="N0" />	
        <argument value="N1" />
    </process> 
    <process host="B1" function="SB_batch">
        <argument value="N2" />	
    </process> 
    <process host="N0" function="SB_node"/>
    <process host="N1" function="SB_node"/>
    <process host="N2" function="SB_node"/>
</platform_description>
\end{verbatim}}


\subsection{Jobs files}
\label{subsec:jobsDESC}

Jobs can be defined using several models. The only one tested for now
in Simbatch is the \verb+wld+ one. You can add more job parsers by
implementing you own plugin.

\subsubsection{wld files}
\label{subsubsec:wld}

The \verb+wld+ parser is implemented within a plugin. Each job is
defined on a single line of the file. The jobs in this kind of file
are described as follows (in this order):

\begin{description}
  \item [user\_id] an ID of each job given by the user (it is NOT the
    id of the user submitting the job)
  \item [submit\_time] the time where the job is submitted by the client
  \item [run\_time] the duration of the job
  \item [input\_size] the size of the data transmitted from the client
    to the batch when submitting the job
  \item [output\_size] the size of the data transmitted from the batch
    to the client when the job is completed
  \item [wall\_time] the time the user requested for the batch submission
  \item [nb\_procs] the number of processors needed by the task
  \item [service] the type of the job (for example type 1 is a matrix
    pultiplication)
  \item [priority] the priority of the job.
\end{description}

\subsubsection{swf files}
\label{subsubsec:swf}

The plugin to parse \verb+swf+ files is created but is not
tested. Please do not use it for now.


%% basic example
%%\section{A basic example}

\subsection{Main program}

When you create a program using simbatch, you main function is defined
just as in Simgrid. First, you need to initialize Simbatch and
Simgrid. Then, you have to register functions that Simgrid will
use. Simbatch proposes two basic functions: SB\_Batch that represents
the batch system to simulate, and SB\_Node that represents a node of
the cluster. SB\_client has to be defined by the user.

Programs written with simbatch must be called likewise:
\begin{verbatim}
program -f simbatch_file.xml
\end{verbatim}

{\small
\begin{verbatim}
int main(int argc, char ** argv) {
    SB_global_init(&argc, argv);
    MSG_global_init(&argc, argv);
 
    MSG_set_channel_number(10000);
    
    /* The client that submits requests */
    MSG_function_register("SB_client", SB_client);
    /* The batch */
    MSG_function_register("SB_batch", SB_batch);
    /* Node of the Cluster */
    MSG_function_register("SB_node", SB_node);
    
    MSG_create_environment(SB_get_platform_file());
    MSG_launch_application(SB_get_deployment_file());
    
    /* Call MSG_main() */
    MSG_main();
    
    /* Clean everything up */
    SB_clean();
    MSG_clean();

    return EXIT_SUCCESS;
}
\end{verbatim}
}

\subsection{The client part of the program}

The client reads jobs from a file and submits it to the batch. 

{\small
\begin{verbatim}
int SB_file_client(int argc, char ** argv) {
    m_host_t sched = NULL;
    xbt_fifo_t bag_of_tasks = NULL;

    if (argc!=5) {
	fprintf(stderr, "Client has a bad definition\n");
	exit(1);
    }
    
    sched = MSG_get_host_by_name(argv[4]);
    if (!sched) {
	fprintf(stderr, "Unknown host %s. Stopping Now!\n", argv[4]);
	exit(2);
    }

    bag_of_tasks = get_jobs("./file.wld");
    
    /* Now just send the job at time to the scheduler */
    if (xbt_fifo_size(bag_of_tasks)) {
	double time = 0;
	job_t job = NULL;
	while ((job=(job_t)xbt_fifo_shift(bag_of_tasks))) {   
	    MSG_process_sleep(job->submit_time - time);
	    job->source = MSG_host_self();
	    MSG_task_put(MSG_task_create("SB_TASK", 0.0, 0.0, job), 
			 sched, CLIENT_PORT);
	    time = job->submit_time;
	}
	xbt_fifo_free(bag_of_tasks);
    }
    return EXIT_SUCCESS;
}
\end{verbatim}
}


%%simbatch.xml

%% Types of tasks we can send to SB_batch
%%\section{SB\_batch}



%% plugins (schedulers)

%% plugins (parsers)

%% platform description

\bibliographystyle{plain}
\bibliography{simbatch}

\end{document}
