\section{Installing Simbatch}
\label{sec:installation}

\subsection{Dependencies}
\label{subsec:dependencies}

To compile Simbatch, you will need to have:

\begin{itemize}
  \item a *nix environment (Simbatch was tested on Linux and Mac OS
    but should work on any *nix environment)
  \item simgrid >= 3.1 (Simbatch is build over simgrid so it is
    obviously a dependency)
  \item gcc >= 4.0 (to compile the project)
  \item cmake >= 2.4 (to create the makefiles)
  \item libxml2 (used to parse input files)
\end{itemize}

\subsection{Compiling and installing Simbatch}
\label{subsec:compilation}

To install Simbatch, the first thing you need to do is to download the
sources. you can go on \url{http://graal.ens-lyon.fr/simbatch} to
download an old release. If you want a more up-to-date version, you
need to download the sources from the Simgrid svn in the contrib
version. Then, create a build directory to work with cmake. then you
can compile and install Simbatch.

\verb+ccmake+ is used instead of cmake because it is a graphical
version of the tool. The first screen you will see will tell you that
the cache is empty. This is normal, so just quit by hitting \emph{e}.
On the next screen, you can change different options. The options will
be details in section \ref{subsec:options}. Hit the \emph{c} key to
test if all needed packages are present. if everything is OK, you can
press \emph{g} to generate the makefiles. Finally, just compile as
usual with make and make install.

{\small
\begin{verbatim}
$>svn checkout svn://scm.gforge.inria.fr/svn/simgrid/contrib/Simbatch
[...]
$>cd Simbatch
$>mkdir build
$>cd build
$>ccmake ..
$>make
[...]
$>make install
[...]
$>
\end{verbatim}}

\subsection{Cmake options}
\label{subsec:options}

When you are in the \verb+ccmake+ GUI, the options you may use are:

\begin{description}
  \item [CMAKE\_INSTALL\_PREFIX] lets you choose the installation
    folder. It is not used currently.
  \item [DEBUG] activates some flags and prints more informations
  \item [GANTT] activates the
    paj{\'e}\footnote{\url{http://www-id.imag.fr/Logiciels/paje/}}
    output to visualize gantt charts. It does not work well for now.
  \item [LIBRARY\_OUTPUT\_PATH] lets you choose where the libraries
    will be put. If you let it empty, the libraries will be places in
    the lib directory where you downloaded the sources.
  \item [LOG] creates a file when executing an example with what
    appens in the batch. For example, it can display: task 1 arrived
    on batch at time 10.
  \item [NDEBUG] does something
  \item [OPT] does something
  \item [OUTPUT] is used to obtain the result of a batch in a
    file. The file contains the list of jobs the batch executed, their
    arrival time, when they started, the completion time and the
    diration.
  \item [PARANOID] activates a lot of compilation flags.
  \item [SIMGRID\_HOME] specifies the folder where simgrid is.
  \item [VERBOSE] displays more informations during an execution.
  \item [WARN] activates the basic warning flags for compilation.
\end{description} 

\subsection {Testing the installation}
\label{subsec:testing}

Go in example/batch and try to run the example. If there is no error,
your installation is ready.

{\small
\begin{verbatim}
$>cd Simbatch/example/batch
$>make
$>./batch -f simbatch.xml
[...]
$>
\end{verbatim}}
