\section{Input files}
\label{sec:input}

Several types of files can be taken as input of a simbatch
program. Every program that you run needs to have at least 3
files. The Simbatch file, and two files used by simgrid. All these
files are described in section\ref{subsec:description}.

Simbatch can also read \verb+wld+ and \verb+swf+ files. These files
are described in section \ref{subsec:jobsDESC}. In those files, jobs
that are executed are described.

\subsection{Description files}
\label{subsec:description}

The description files are describing the batchs, the hardware and the
deployment of the processes on the hardware.

\subsubsection{Simbatch file}
\label{subsubsec:SBfile}

This XML file is quite simple. First, you give the simgrid files
(deployment and platform). Then, the batchs are described. You can
describe more than one batch. The example bellow displays a simualtion
with 2 batchs. The first one uses the Conservative Backfilling
algorithm to allocate resources. It has an internal load (so we need
to give the parser of the load). The second batch uses First Come
First Serve to allocate resources to jobs, and is dedicated (no load
is given).

{\small
\begin{verbatim}
<config>  
    <global>
        <file type="platform">platform.xml</file>
        <file type="deployment">deployment.xml</file>
    </global>
    <batch host="Batch0">
        <plugin>libfcfs.so</plugin>
        <load>1.wld</load>
        <parser>libwld.so</parser>
        <priority_queue>
            <number>3</number>
        </priority_queue>
    </batch>
    <batch host="Batch1">
        <plugin>libcbf.so</plugin>
        <priority_queue>
            <number>3</number>
        </priority_queue>
    </batch>
</config>
\end{verbatim}}

\subsubsection{Platform file}
\label{subsubsec:PFfile}

The platform file describes the hardware of the simulation. It is the
old version of platform description in Simgrid. The example below
describes a platform with 2 batchs. The first batch has 2 nodes and
the second one has just one node. The description is done by
describing CPUs, network links capacities and routes.

{\small
\begin{verbatim}
<?xml version='1.0'?>
<!DOCTYPE platform_description SYSTEM "surfxml.dtd">
<platform_description version="1">
  <!-- Cluster 0 -->
  <cpu name="B0" power="3865000"/>
  <cpu name="N0" power="3865000"/>
  <cpu name="N1" power="3865000"/>

  <network_link name="LB0" bandwidth="498000" latency="0.0"/>
  <network_link name="BW0" bandwidth="1000000" latency="0.0"/>
  <network_link name="BW1" bandwidth="1000000" latency="0.0"/>
  
  <route src="B0" dst="B0"><route_element name="LB0"/></route>
  <route src="B0" dst="N0"><route_element name="BW0"/></route>
  <route src="B0" dst="N1"><route_element name="BW1"/></route>
  <route src="N0" dst="B0"><route_element name="BW0"/></route>
  <route src="N1" dst="B0"><route_element name="BW1"/></route>

  <!-- Cluster 1 -->
  <cpu name="B1" power="3865000000"/>
  <cpu name="N2" power="3865000000"/>
  
  <network_link name="LB1" bandwidth="4980000" latency="0.0"/>
  <network_link name="BW2" bandwidth="1000000" latency="0.0"/>
  
  <route src="B1" dst="1"><route_element name="LB1"/></route>  
  <route src="B1" dst="N2"><route_element name="B2"/></route>  
  <route src="N2" dst="B1"><route_element name="B2"/></route>
</platform_description>
\end{verbatim}}

\subsubsection{Deployment file}
\label{subsubsec:DEPfile}

The deployment file describes the Simgrid processes that will be used,
and where each process will be executed. For example, the process
\verb+SB_batch+ will run on the host B0 and will take 1 argument. The
arguments have to contain the list of nodes of the batch.

{\small
\begin{verbatim}
<?xml version='1.0'?>
<!DOCTYPE platform_description SYSTEM "surfxml.dtd">
<platform_description version="1">
    <process host="Client" function="SB_client" >
        <argument value="B1" />
    </process>
    <process host="B0" function="SB_batch">
        <argument value="N0" />	
        <argument value="N1" />
    </process> 
    <process host="B1" function="SB_batch">
        <argument value="N2" />	
    </process> 
    <process host="N0" function="SB_node"/>
    <process host="N1" function="SB_node"/>
    <process host="N2" function="SB_node"/>
</platform_description>
\end{verbatim}}


\subsection{Jobs files}
\label{subsec:jobsDESC}

Jobs can be defined using several models. The only one tested for now
in Simbatch is the \verb+wld+ one. You can add more job parsers by
implementing you own plugin.

\subsubsection{wld files}
\label{subsubsec:wld}

The \verb+wld+ parser is implemented within a plugin. Each job is
defined on a single line of the file. The jobs in this kind of file
are described as follows (in this order):

\begin{description}
  \item [user\_id] an ID of each job given by the user (it is NOT the
    id of the user submitting the job)
  \item [submit\_time] the time where the job is submitted by the client
  \item [run\_time] the duration of the job
  \item [input\_size] the size of the data transmitted from the client
    to the batch when submitting the job
  \item [output\_size] the size of the data transmitted from the batch
    to the client when the job is completed
  \item [wall\_time] the time the user requested for the batch submission
  \item [nb\_procs] the number of processors needed by the task
  \item [service] the type of the job (for example type 1 is a matrix
    pultiplication)
  \item [priority] the priority of the job.
\end{description}

\subsubsection{swf files}
\label{subsubsec:swf}

The plugin to parse \verb+swf+ files is created but is not
tested. Please do not use it for now.
